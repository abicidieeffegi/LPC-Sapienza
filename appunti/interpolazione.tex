\section{Interpolazione di polinomi}
È utilizzata per approssimare una qualche funzione $f(x)$ con un polinomio $p_n \in \mathbb{R}[x]_{\leq n}$
Un altro uso comune è, dato un insieme di punti $\{(x_i, y_i)\}_{i=0}^n$, trovare un polinomio 
che meglio approssima questi dati (detti nodi), ovvero un polinomio tale che $p_n(x_i) = y_i$.

\paragraph{Interpolazione di Lagrange}
Data una griglia $X = \{x_0, x_1,\dots , x_n\}$ con $x_i < x_{i+1} \forall i \in \{0, 1,\dots , n\}$
Definiamo, sulla griglia $X$ gli $n+1$ polinomi di base di Lagrange come:
$$
l_i(x) = \prod_{j \neq i} \frac{x-x_j}{x_i-x_j}\ i \in \{0, 1,\dots , n\}
$$
Tali polinomi sono tutti di grado n.
Notiamo che:
$$
l_i(x_j) = \delta_{ij} =
\left\{
  \begin{array}{ll}
    0 & j \neq i\\
    1 & j=i
  \end{array}
\right.
$$
Si dimostra inoltre che tali polinomi sono una base dello spazio vettoriale $\mathbb{R}[x]_{\leq 2}$.\\
Il polinomio di interpolazione di Lagrange sarà:
$$
p_n(x) = \sum_{i=0}^n y_i \cdot l_i(x)
$$
Si dimostra inoltre che tale polinomio è unico.

\paragraph{Errore}
Si considera l'errore come $e_n(x) = f(x) - p_n(x)$, $x = x_i \implies e_n(x) = 0$
Inoltre si può dimostrare che:
$$
e_n(x) = \prod_{i=0}^n (x-x_i)\frac{f^{(n+1)}(\xi)}{(n+1)!}
$$
Si noti che non è vero che $e_n \xrightarrow{n \to \infty}0$.

\paragraph{Fenomeno di Runge}
Avviene quando si approssimano funzioni a gradini.
Per limitare la funzione nodale si scelgono i nodi di Chebychev:
$$
x_i = \frac{a+b}{2} + \frac{b-a}{2} \cdot \cos\left(\frac{2k+1}{2(n+1)}\pi \right)
$$
