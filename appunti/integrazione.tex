\section{Integrazione numerica}
$$
  I = \int_{a}^{b} f(x) \,dx = F(b) - F(a)
  \land F \in \{f: [a, b] \to \mathbb{R}\}
$$
Formule di quadratura numerica:
$$
  I \approx Q(f) = \sum_{k=1}^M \omega_k f(x_k)
$$

\subsection{Regola dei rettangoli}
Considero $f: [a, b] \to \mathbb{R}$ e divido $[a, b]$ in $M$ sottointervalli 
di uguale ampiezza $h = \frac{b-a}{M}$:
$$
  \Omega_k = [x_{k-1}, x_k]
$$
Da cui segue che:
$$
  I = \int_{a}^{b} f(x) \,dx = \sum_{k=1}^M \int_{\Omega_k} f(x) \,dx
$$
Approssimiamo $I$ come:
$$
  I \approx \sum_{k=1}^{M} f(x_{k-1}) \cdot h = h \sum_{k=1}^{M} f(x_k)
$$

\paragraph{Convergenza}
Dai nostri metodi speriamo che:
$$
  Q(f) \xrightarrow{h \to 0} I
$$
Ovvero che:
$$
  err(h) = |I - Q(f)| \to_{h \to 0} 0
$$
Con $err(h) \leq C(a, b; f) \cdot h$
Ovvero quando dimezzo h dimezzo err(h). 
Inoltre, la formula dei rettangoli ha ordine di accuratezza 1.
La formula dei rettangoli è una formula ad un punto.

\subsection{Regola del punto medio}
È analoga alla regola dei rettangoli, ma anziché considerare come altezza $f(x_k)$, 
considero $f(c_k)$, con $c_k = \frac{x_k + x_{k-1}}{2}$. 
Quindi:
$$
  Q(f) = h \sum_{k=1}^{M} f(c_k)
$$
Questo metodo ha ordine di accuratezza 2:
$$
  err(h) \leq C(a, b; f) \cdot h^2
$$

\subsection{Formula del trapezio}
Considero le medesime premesse dei casi precedenti. 
Costruisco dei trapezi su ogni intervallo con basi $f(x_k)$ e $f(x_{k-1})$.
$$
  Q(f) = \frac{h}{2} \sum_{k=1}^{M} [f(x_{k-1}) + f(x_k)]
$$
La formula del trapezio è accurata di ordine 2. 
La formula del trapezio è a due punti.

\subsection{Formula di Gauss}
Considero i seguenti punti:
\begin{align*}
  \gamma_{k1} = x_{k-1} + (1 - \frac{1}{\sqrt{3}})\frac{h}{2} \\
  \gamma_{k2} = x_{k-1} + (1 + \frac{1}{\sqrt{3}})\frac{h}{2}
\end{align*}
Da cui segue che:
$$
  Q(f) = \frac{h}{2}\sum_{k=1}^{M}[f(\gamma_{k1}) + f(\gamma_{k2})]
$$
Si può dimostrare che il metodo è accurato di ordine 4:
$$
  err(h) \leq C(a, b; f) \cdot h^4
$$

\subsection{Formula di Simpson}
Considero $f(x_{k-1})$, $f(c_k)$ e $f(x_k)$. 
Su ogni $\Omega_k$ costruisco quindi la parabola che interpola $f(x)$. 
$$
  Q(f) = \frac{h}{6}\sum_{k=1}^{M} [f(x_{k-1}) + 4f(c_k) + f(x_k)]
$$
Questa è una formula a tre punti con accuratezza di ordine 4.

\subsection{Montecarlo}
Al contrario delle formule precedenti, questa è una formula stocastica (non deterministica).
Costruisco un rettangolo $R = [a, b] \times [0, \max\limits_{x \in [a,b]} f(x)]$. 
Prendo dei punti casuali nel rettangolo, $K$ cadranno sotto la funzione e $N$ sono i totali.
$$
  Q(f) = \frac{K}{N} A_R
$$
Dove $A_R$ è l'area del rettangolo. 
Montecarlo è un metodo molto lento, ma il suo ordine di accuratezza non decresce 
in dimensioni superiori a 1.
